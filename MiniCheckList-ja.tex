\documentclass[11pt,a4j]{jsarticle}
\usepackage{url}

\begin{document}
\title{卒業論文・修士論文自己チェックリスト}
\author{後藤 祐一 \thanks{埼玉大学大学院 理工学研究科 数理電子情報部門 情報領域 先
端情報システム工学研究室}\\ gotoh@aise.ics.saitama-u.ac.jp}
\maketitle

\begin{abstract}
 卒業論文や修士論文の第0稿から第1稿にする際に何をチェックすべきかをまとめ
 た。このチェックリストにしたがい論文の頭からお尻まで推敲を繰り返せば、
 論文の書式や論文の書き方、そして、日本語の使い方についてある程度改善さ
 れた状態にすることができる。
\end{abstract}

\setcounter{section}{-1}

\section{本チェックリストの使い方}

論文執筆の際の有用な助言に「とにかく、どんなに粗末なものでも良いので頭からお尻まで論文を一通り書き上げなさい」というものがある。しかし、この「とりあえず書き上げたもの」は、論文とみなすことはできない。この段階では第1稿未満の第0稿である。

論文指導をしてくれている先輩や教員にこの第0稿を渡しても、役に立つ助言や指導を受けることはできない。なぜならば、論文の内容ではなく、論文の書式や論文の書き方、そして、日本語の使い方についての指摘に時間が費やされてしまうからである。

本チェックリストは、「とりあえず書き上げた」第0稿を、論文の書式や論文の
書き方、そして、日本語の使い方についてある程度改善された第1稿にバージョ
ンアップさせるものである。チェックリストはいくつかの節から構成されている。
各節ごとにどの点に注目して論文を推敲するのかが指示されている。論文を7回推敲することで論文の書式や論文の書き方、そして、日本語の使い方についてある程度改善された状態にすることができる。

このチェックリストを活用し、まともな論文指導が受けられる原稿を作り上げて
欲しい。また、大学院生が同級生あるいは後輩への論文指導を行う際にどの点に注意して
指導をしたら良いかの参考にも本チェックリストを利用して欲しい。

\section{1回目のチェック}

1回目のチェックは、主に書式や形式に関わる点に関してのミスを潰す。書式や
形式に関するミスが残っていると、指導をする側としては有意義な助言や本質的
な改善点を指摘できない。


\subsection{書式}

経験的に書式が整っていない文書は、内容も粗末であることが多い。自分が書い
た文書を相手に読んでほしいのであれば指定された書式どおりに文書を作成する
必要がある。

卒論・修論指導の立場から言えば、書式を守るという行為は論文執筆経験が不要
な行為であり、時間をかけ、注意深く自分の原稿を見直せば必ず達成できる行為
であると認識している。つまり、書式が守れないということは、「執筆者は時間
をかけていない」あるいは「執筆者は自分の論文を大事にしていない」と判断さ
れてもしょうがないということである。執筆者すら手間をかけていない論文なら、
当然、論文指導を行う側は手間をかけて指導しない。

ぜひ、書式のミスは自己チェックで潰して欲しい。

\begin{flushleft}
 {\bf チェックリスト}
\end{flushleft}
\begin{list}%
 {□} %default label
 {} %formatting parameter
 \item 学科指定の卒論・修論TeXスタイルファイルを使っている。
 \item 用紙サイズがA4縦である。
 \item 謝辞を除き、常体(〜だ。〜である)で書かれている。
 \item 英数字は半角文字である。
 \item 句読点は、全角のカンマ(,)とピリオド(.)か、全角のカンマ(,)
       と丸(。)か、全角の点(、)と丸(。)のどれかで統一されている。なお、
       情報処理学会論文誌の句読点は全角のカンマ(,)とピリオド(.)と指
       定されている \cite{IPSJ_HowToWrite_ver7.1} ので、程研究室もそれに
       準ずる。
 \item 英文中や数式中のカンマは半角のカンマ(,)を使っている
       \cite{IPSJ_HowToWrite_ver7.1}。
 \item 括弧は全角の「(」と「)」を使っている。ただし、英文中、図表見出し、参
       考文献の書誌データでは半角の「(」と「)」をつかう
       \cite{IPSJ_HowToWrite_ver7.1}。
 \item スペースはすべて半角スペースを使っている
       \cite{IPSJ_HowToWrite_ver7.1}。
 \item 英数字で構成される単語の前後には半角スペースを入れている
       \cite{IPSJ_HowToWrite_ver7.1}。
 \item {\bf 単語の省略表現に使うピリオドの後ろは半角スペース1つである。}
       英語の組版では通常文末に半角スペースを二つ挿入する。LaTeXを使う場
       合、LaTeXが勝手にピリオドの後ろに半角スペースを2ついれてくれる。
       一方、「Dr.」や「Mr.」のような省略を表すピリオドの後ろにも半角ス
       ペースをいれてしまうことがあるため、明示的に半角スペース1つである
       ことを示す必要がある。バックスラッシュ+半角スペースで半角スペース
       1つを明示的に指示できる。
 \item カタカナは全角カタカナを使っている \cite{IPSJ_HowToWrite_ver7.1}。
 \item 引用符では開きにバッククォート2つ(``)を使い、閉じにはシングル
       クォート('')2つを使っている \cite{IPSJ_HowToWrite_ver7.1}。
 \item 表や図、数式や変数名が枠からはみだしていない(LaTeXの場合は
       OverfullやUnderfullをおこしていない)
       \cite{IPSJ_HowToWrite_ver7.1}。
 \item 表紙にはページ番号が表示されていない。
 \item 概要、謝辞、目次、表目次、図目次のページ番号がローマ数字である。
 \item 1章以降のページ番号がアラビア数字である。
 \item 概要、謝辞、表目次、図目次、参考文献には章番号がついていない。
 \item 付録の章番号はアルファベットである($A, B, C, \ldots$)。
\end{list}

\subsection{表紙}

以下のチェックリストは埼玉大学工学部情報システム工学科および埼玉大学大学
院理工学研究科数理電子情報部門情報システムコースに特化したものであること
に注意して欲しい。

\begin{flushleft}
 {\bf チェックリスト}
\end{flushleft}
\begin{list}%
 {□} %default label
 {} %formatting parameter
 \item 右肩の論文番号が学科の指定どおりに記載されている。
 \item 論文のタイトルが記載されている。
 \item タイトルが複数行にわたるとき、文節の切れ目で改行されている。
 \item 指導教員名が記載されている。
 \item 「指導教官」ではなく「指導教員」となっている。
 \item 職位が正確にかかれている。2010年2月現在は、大学教員の職位は、教授、
       准教授、講師、助教、助手となっている。
 \item 指導教員名の姓と名が半角スペースで区切られている
       \cite{IPSJ_HowToWrite_ver7.1}。
 \item 提出日付が正しい。
 \item 所属名が正しい。
 \item 学籍番号と名前が記載されている。
 \item 氏名の姓と名が半角スペースで区切られている
       \cite{IPSJ_HowToWrite_ver7.1}。
 \item 研究室名と住所が記載されている。
 \item 住所が正しい。
\end{list}

\subsection{謝辞}

謝辞には、2つのことを書く。第一に研究を進める上でお世話になった人に対す
る感謝の言葉、第二に学部あるいは大学院前期課程(修士課程)の締めくくりと
して、自分が卒業/修了するまでにお世話になった人たちへの感謝の言葉である。

基本的に礼儀正しくかつ丁寧に書くことを除き、謝辞には決まった形式はない。ただ
し、多くの人は生まれて初めて謝辞を書くことになると思うので、研究室や学科/
専攻の先輩方の謝辞を参考にすることを強くおすすめする。

\begin{flushleft}
 {\bf チェックリスト}
\end{flushleft}
\begin{list}%
 {□} %default label
 {} %formatting parameter
 \item 謝辞に登場する人物がフルネームで記載されている(ただし、「両親」
       「先輩諸氏」「先生方」などの集団の属性に対する表現はそのままで良
       い)。
 \item 教員の職位が正しい。2010年2月現在は、大学教員の職位は、教授、
       准教授、講師、助教、助手となっている。
 \item 教員ではない博士号持ちの研究協力者の敬称が「氏」ではなく「博士」
       になっている。
 \item 教員ではない研究協力者の敬称が「氏」になっている。
\end{list}

\subsection{目次}

専門書、学位論文(博士論文、修士論文、卒業論文)などを読むとき、多くの人
は、タイトル、序言(巻頭言、はじめに)、目次、索引を読み、この本を読むべ
きかどうか、どの部分を読むかを決める。

よって、目次の見やすさや章/節/小節タイトルの適切さはとても重要である。
また、目次だけで内容を把握できるようにするため、できる限り造語(論文内で
独自に定義した言葉)や略語を使わないように心がける必要がある。

\begin{flushleft}
 {\bf チェックリスト}
\end{flushleft}
\begin{list}%
 {□} %default label
 {} %formatting parameter
 \item 目次には以下の項目が記載されている。
       \begin{itemize}
	\item 概要
	\item 謝辞
	\item 目次
	\item 表目次
	\item 図目次
	\item 1章〜終章(章番号あり、アラビア数字)
	\item 参考文献
	\item 付録(章番号あり、アルファベット)
       \end{itemize}
 \item ある章(節)において節(小節)は必ず複数個ある。たとえば、2章には
       2.1しか存在しない、あるいは、2.1節には2.1.1しか存在しないというこ
       とがない。 (文献 \cite{R.M.Lewis_HowToWrite_Ja_04} の第29章「パラ
       レリズムを論文にとりいれる」を参照のこと。)
 \item コラムのような(たとえば「〜とは」や「〜とは何か?」のような)章/節/小節タ
       イトルは存在しない。
 \item 造語が章/節/小節タイトルに含まれていない。
 \item 略語が章/節/小節タイトルに含まれていない。
\end{list}

\subsection{参考文献リスト}

参考文献としているが、実際には引用につかった文献のことを指している。自分
が研究を進める際に参考にした文献を列挙する場所ではない。
参考文献は二つの役割を持つ。一つは、他者の主張あるいは他者の研究成果と
自分の主張あるいは自分の研究成果を明確に分ける役割である。二つめは、読者
の理解の手助け、あるいは論文の読みやすさのために、詳細や根拠となる実験お
よび実験結果を他の文献に委ねる役割である。

よって、参考文献リストでもっとも重要なことは、引用文献を確実に特定できる
ということである。次に重要なことはその文献をちゃんと入手できる(可能性が
ある)ということである。公開されていない文献や入手不可能な文献を参考文献
として用いてはいけない。

\begin{flushleft}
 {\bf 本文中の引用に対するチェックリスト}
\end{flushleft}
\begin{list}%
 {□} %default label
 {} %formatting parameter
 \item 他人の主張や研究成果についてはすべて参考文献を引いている。
 \item 参考文献の引用は文中にある。すなわち「〜である。 [xx]」ではなく「〜
       である [xx]。」となっている。
 \item 何を引用しているのかが誤解されないようになっている。詳しくは文献
       \cite{Sakai_HowToWrite_02}の良くない引用の例を参照のこと。
 \item 論文中で人名を挙げて引用する場合には敬称をつけていない。つまり、
       「〜のツールを開発した**先生は、…」「**氏の理論では、…」で
       はなく、「〜のツールを開発した** [xx] は、…」「** [xx] の理
       論では、…」とする。
 \item 引用で文が終わる場合は句点は引用符の中につける。たとえば「``〜で
       ある''。」ではなく、「``〜である。''」とする。
 \item 引用について参考文献を示したい場合に参考文献番号が引用符の外にあ
       る。つまり、「``〜である。'' [xx]」としてある。
 \item (TeXの場合)クロスリファレンスがちゃんと表示されている
       ([?]みたいになっていない)。

\end{list}


\begin{flushleft}
 {\bf 参考文献リストに対するチェックリスト}
\end{flushleft}
当然、例外事項がありえるのでチェックリストに反している文献に関しては指導
教員に問い合わせること。
\begin{list}%
 {□} %default label
 {} %formatting parameter
 \item 参考文献リストに記載されている文献は、すべて論文内で引用した文献
       である。
 \item 参考文献リストに記載されている文献は、すべて公開されている文献お
       よび入手可能な文献である。卒業論文、修士論文においては同学科/専
       攻/コースの過去の卒業論文や修士論文を参考文献として良い(入手可
       能なため)。
 \item 文献の発行(発表)年月の詳しさが統一されている。年と年月が混ざっ
       ていない。
 \item 参考文献として挙げたWebサイトは、引用した内容がそのWebサイト以外
       に存在しないから掲載している。つまり、本、論文で代用可能でない。
 \item 文献リストの構成要素(著者名、タイトル、編者、書名、巻、号、ペー
       ジ番号、出版社、発行年月など)の並びが統一されている。少なくとも、
       同じ種類(本、会議録掲載論文、雑誌掲載論文)ごとに構成要素の並び
       順が統一されている。
\end{list}

\begin{flushleft}
 {\bf 程研究室用参考文献リストに対するチェックリスト}
\end{flushleft}
Webページにある例を真似てつくること。特に間違えやすい点だけをリストアッ
プする。
\begin{list}%
 {□} %default label
 {} %formatting parameter
 \item 英語文献が先、和文献は後になっている。
 \item 文献リストの順番は個人の姓あるいは組織名で辞書順(英語文献なら
       「A」〜「Z」、和文献なら「あ」〜「ん」)である。
 \item 著者名がフルスペルで書かれている。
 \item 英語論文の著者名のファミリーネームはすべて大文字である。
 \item 学術雑誌掲載論文の場合は「著者名: 論文タイトル, 雑誌名, 巻, 号,
       ページ番号, 発行年月.」という構成になっている。
 \item 「Vol.」や「No.」などの略語を表すピリオドの後ろは半角スペース1つ
       である(LaTeXでは通常はピリオドの後に半角スペース2つが入る。半角
       スペース1つにする場合には、バックスラッシュ+半角スペースとす
       る)。
 \item 雑誌名は省略しない。
 \item 本の場合「著者名、訳者名: 本のタイトル 雑誌名,出版社, 発行年月.」
       という構成になっている。
 \item 本の章や本に収録された論文の場合「著者名: 論文/章のタイトル, 編
       集者&本のタイトル, ページ番号, 出版社, 発行年月.」という構成になっ
       ている。
 \item 英語文献において、編集者&本のタイトルは「in 編集者名 (Ed./Eds.), ``本のタイト
       ル,''」という形式にする。編集者名のファーストネームはイニシャルに
       する。編集者が一人の場合は (Ed.) 、二名以上の場合は (Eds.)とする。
 \item 論文集収録論文の場合「著者名: 論文のタイトル, 論文集名, ページ番
       号, 会議の開催場所, 発行年月.」という構成になっている。
 \item 英語論文の論文集名は「Proceedings of 会議名」という形式になってい
       る。
 \item 英語論文の会議の開催場所は「都市名, 国名」となっている。国名は正
       式名称でなくてよい。たとえば、ChineやUSA、UKなど。
\end{list}


\section{2回目のチェック}

2回目のチェックは、図、表、例題、数式、箇条書きなどの形式について
チェックを行う。

\subsection{図、表、例題}

図や表、例題を論文に載せる目的は、読者によりわかりやすく実験結果や調査結
果、主張やアイデアなどを伝えることにある。よって、この目的に沿わない使い
方をしてはならない。

\begin{flushleft}
 {\bf チェックリスト(1回目)}
\end{flushleft}
1回目の見直しで形式的なミスをすべて潰すこと。
\begin{list}%
 {□} %default label
 {} %formatting parameter
 \item タイトルの位置は表の上、図の下にある(例題については分野の習慣に
       準ずる)。
 \item 本文中でもタイトル中でも図と表の表記が統一されている。たとえば、
       本文中で図1ならばタイトル中でも図1である。本文中で図1、タイトル中
       でFig.\ 1となってはいけない。
 \item 図や表がページからはみ出してない(LaTeXならOverfullやUnderfullを
       起こしていない)。
 \item 図中や表中の文字がちゃんと読める(印刷でつぶれていない)。
 \item 図中や表中の文字が文字化けしていない。
 \item 図中や表中の数式がちゃんとイタリックになっている。
 \item 図は白黒印刷でもちゃんと理解できる。
 \item 図、表、例題がそれ単独で見ても大体理解できるようになっている。
       \begin{itemize}
	\item 図、表、例題にタイトル(キャプション)がちゃんとついている。
	\item グラフに凡例がついている。
	\item 図中、表中の数字の単位がわかる。
	\item 図中の記号や図形の意味がちゃんと説明されている。
	\item 図中の図形の種類、色、線の種類(実線、破線など)の意味が直
	      観的に理解できる。
       \end{itemize}
 \item 図や表や例題が必ず本文中で説明されている。
 \item 図、表、例題が本文中の説明のすぐ近くにある(離れていたとしても高々
       次ページが限度)。
 \item 有効数字がちゃんと検討されている。
\end{list}

\subsection{箇条書き}

概念の抽象度や説明の詳しさが同程度の事柄をわかりやすく説明するのに箇条書
きは非常に便利である。ただし、箇条書きを使う目的も図や表と同じように、筆
者の主張や説明をよりわかりやすく読者に提供するためにあるということを忘れ
ないようにする。箇条書きから筆者の主張や説明したい事柄を類推させてはいけ
ない。

\begin{flushleft}
 {\bf チェックリスト(1回目)}
\end{flushleft}
1回目の見直しで形式的なミスをすべて潰すこと。
\begin{list}%
 {□} %default label
 {} %formatting parameter
 \item 数字なし箇条書き(LaTeXの場合は itemize 環境)で列挙されているも
       のの表現はすべて同じである。つまり、主語や時制、文なのか非文なの
       か、英語ならば、名詞なのか動名詞なのか不定詞なのかが同じである。
 \item 数字あり箇条書き(LaTeXの場合は enumerate 環境)で列挙されている
       ものは、列挙順に意味がある。
 \item 箇条書きの入れ子は、概念の抽象度や説明の詳しさが同レベルの事柄が
       2個以上あるときだけ使われている。 詳しくは、文献
       \cite{R.M.Lewis_HowToWrite_Ja_04}の第29章「パラレリズムを論文に取
       入れる」を参照のこと。
       \begin{itemize}
	\item たとえば、こういう1つだけの箇条書きを使っていない。
       \end{itemize}
 \item 箇条書きのみで構成された段落が存在しない。
\end{list}

\subsection{数式}

自然言語での説明で曖昧な事柄でも、数式を使えば簡単にかつ厳密に説明するこ
とができる。一方で、卒業論文や修士論文の想定読者は扱うテーマの専門家では
ないため、ある分野では有名な数式、あるいは常識的な変数名を知らない可能性
がある。よって、国際会議論文集や学術雑誌に掲載する論文よりも丁寧に説明す
る必要がある。

\begin{flushleft}
 {\bf チェックリスト}
\end{flushleft}
\begin{list}%
 {□} %default label
 {} %formatting parameter
 \item 重要な数式にはすべて番号がふられている。
 \item 重要な数式についてはそれが何を表すのかを本文中でちゃんと説明して
       いる。
 \item 数式もしくは文中の変数はイタリックで記載されている(LaTeXであるな
       らば、ちゃんと数式環境を用いている)。
 \item 数式に登場する変数($\sum$などで使われるカウンター用途の変数を除
       く)はすべてちゃんと本文中で説明されている。
 \item 論文中では変数は可能な限り一意に使われている。つまり、ある場所で
       は「論理式$A$が…」、別の場所で「集合$A$が…」、別の場所で「ソー
       スコードの行数を$A$とする」というような使い方をしていない。
 \item プログラムや物理などで典型的に使われる変数名に別の意味を与えてい
       ない。たとえば、重力加速度を表すのに$g$を使わず$x$を使う、速さを
       表す変数に$v$を使わず$a$を使うなどをしてない。
\end{list}

\subsection{アルゴリズム、擬似コード}

アルゴリズムの提案を行った場合には必ず擬似コードでそのアルゴリズムを示す
こと。書き方については参考となる文献が見当たらないので暫定的に文献
\cite{Wikipedia.en_Psuedocode}のリンク先を参照のこと。

\begin{flushleft}
 {\bf チェックリスト}
\end{flushleft}
\begin{list}%
 {□} %default label
 {} %formatting parameter
 \item 行番号がついている。
 \item 特定のプログラミング言語に依存していない。つまり、プログラミング
       経験者であるならば理解できる程度の抽象度で書かれている。
 \item 予約語は一目でわかるようになっている。たとえば、if-else-then や
       while、forなどが太字やすべて大文字で強調されている。
 \item 適度に自然言語での説明が入っており詳細過ぎない。
 \item 字下げを用いて制御構文が理解しやすい。
 \item 擬似コード中で未説明の関数や変数、定数については本文中で説明され
       ている。
\end{list}


\section{3回目のチェック}

3回目のチェックは、図、表、例題、箇条書きの内容についてチェックを行う。

\subsection{図、表、例題}

\begin{flushleft}
 {\bf チェックリスト(2回目)}
\end{flushleft}
2回目のチェックでは主に内容についてチェックする。
\begin{list}%
 {□} %default label
 {} %formatting parameter
 \item 図や表のタイトルが何を表した図表なのか具体的にわかるものになって
       いる。
 \item グラフや表に載せている数字やデータは、説明したい事柄に対して多過
       ぎない。
 \item グラフや表に載せている数字やデータは、説明したい事柄に対して少な
       過ぎない。
 \item グラフの種類は説明したい事柄に対して適切である。
 \item 図、表、例題と本文中の説明は食い違っていない。
 \item 図、表、例題から読者が読み取れるものと、自分の主張が食い違ってい
       ない。
 \item 図、表、例題が自分の主張を説明するためにはかかせないものである。
       つまり、惰性で(なんとなく)載せている図、表、例題は存在しない。
 \item 表において不必要な罫線はすべて取り除いてある。
\end{list}

\subsection{箇条書き}

\begin{flushleft}
 {\bf チェックリスト(2回目)}
\end{flushleft}
2回目のチェックでは主に内容についてチェックする。
\begin{list}%
 {□} %default label
 {} %formatting parameter
 \item 箇条書きで列挙されている事柄の概念の抽象度や説明の詳しさはすべて
       同一である。
 \item 筆者の主張や説明したい事柄を箇条書きから読者に読み取らせるように
       はなっていない。つまり、箇条書きが説明したい事柄や筆者の主張を補
       足するものになっている。
 \item 箇条書きにした方が文章で書くよりも、筆者の主張や説明したい事柄を
       よりわかりやすくしている。
\end{list}

\section{4回目のチェック}

4回目のチェックは単語と文に焦点をあててチェックする。

\subsection{単語}

できる限り辞書に準じた用語を使う。また、概要、第1章(はじめに)、本論(2
章〜終章の一つ前)、最終章(おわりに)の4つは独立に読まれるので(多くの
場合「概要→第1章→最終章→本論」の順番で読む)、造語や略語の定義はそれ
ぞれの部分ごとに行なう。

\begin{flushleft}
 {\bf チェックリスト}
\end{flushleft}
\begin{list}%
 {□} %default labels
 {} %formatting parameter
 \item 辞書どおりの意味では使わない、あるいは、一般的に広く受け入れられ
       ている定義が存在しない概念や用語については、必ず初登場時に定義ある
       いは説明をしている。
 \item 情報システム工学科(同じ学科・コースの)4年生が常識として知らない
       ような言葉ならば必ず用語の説明をいれている。
 \item 一般的に使われている言葉を特別な意味で使っていない 。どうしても、
       避けられないならばちゃんと定義し、索引にも載せる。
 \item ある事柄や概念は論文中では常に一つの用語で表現している。論文にお
       いては言い換えは避ける。
 \item 造語を使うときには必ず定義後に使用している。
 \item 略語は、初回使用時に必ずフルスペルを示したのちに使用している。
 \item 過剰な形容詞や修飾語を用いていない。たとえば、「史上空前の…」
       「とても…」「非常に…」など使わない。
\end{list}

\subsection{文}

単語を文法にしたがって並べたものが文である。できるかぎりわかりやすい文を用いる
ようにする。

\begin{flushleft}
 {\bf チェックリスト}
\end{flushleft}
\begin{list}%
 {□} %default label
 {} %formatting parameter
 \item 主語と目的語を省略していない。
 \item 主語と述語がちゃんと対応している。
 \item わかりづらい複文になっていない。主語と述語が1組ずつあるのが単文。
       文中に主語と述語が二組以上あるのが複文。複文は意味が曖昧になりや
       すいのでできる限り単文にする。
 \item 断言を避ける目的で「〜的、〜風、〜性、〜調」を使っていない。
 \item 修飾語、形容詞、副詞の修飾先がはっきりしている。つまり、ある文が
       複数の意味に解釈されない。
 \item 形容詞(長い、重い、早い、など)や形容動詞(綺麗な、新鮮な、など)
       を使うときには量をはっきりさせている。
 \item こそあど言葉を使いすぎていない。
 \item 対応する日本語が存在する限り、カタカナ表現を使っていない。
 \item 意味の曖昧な複合熟語は使っていない(Googleで検索し、1,000件以下な
       らば使ってはいけない)。
 \item 体言止めを使っていない。
 \item 上から目線の書き方をしていない。たとえば、「〜であろう」、「今
       後の課題として〜が挙げられる」のような書き方をしていない。
\end{list}

\section{5回目のチェック}

5回目のチェックでは、文章に焦点をあててチェックする。特に段落ごとの論理
性や段落の構成に着目する。

文を意味のまとまりごとにまとめたのが文章である。事実と事実の解釈、自分の主張や
仮説をごちゃまぜにしないようにする。

\begin{flushleft}
 {\bf チェックリスト}
\end{flushleft}
\begin{list}%
 {□} %default label
 {} %formatting parameter
 \item どれが事実で、どれが主張で、どれが仮説なのかを読者が理解しやすい
       ようにしてある。
 \item 一つのパラグラフ(段落)は一つのトピック(話題や主張)でなりたっ
       ている。
 \item 文章において視点が統一されている。ある文では利用者目線で述べてい
       るのに、次の文は急にシステム目線になり、さらに次の文では利用者目
       線に戻るということが発生していない。
 \item 文章において時制が統一されている。突然、過去時制で自分の行ったこ
       とを説明しているときに、突然、現在時制になったり、未来時制になっ
       て「〜するつもりである」と今後の予定を語りだしたりしない。
 \item 論理の飛躍はない。読んでいて、ひっかかる、あるいは一度後ろに戻っ
       て読み直さないと理解できない文章はない。
 \item 同じ接続詞をすぐ次の文で使っていない。
 \item 説明は、全体から部分、抽象から具体、概略から詳細、過去から未来と
       いう順番で延べられている。
 \item 新しい言葉や概念を定義する際には、必ず一般的な定義を示してから、
       具体的な事柄で説明している。いきなり、「たとえば、〜」などと具体
       例から始めてはいけない。
\end{list}

\section{6回目のチェック}

論文の各章(節)において書かれているべきもの書かれているかどうかに焦点を
当ててチェックする。

\subsection{概要}

文献調査に関して教えられる論文の読み方は次のとおり。まず論文のタイトルを
読み、次に論文の概要を読む。論文の概要を読んで、自分が欲する内容であると
思ったら、第1章(はじめに)と終章(おわりに)を読む。そこまで読んで、より
知りたいことや詳しい内容を知りたかったならば、2章以降を読む。

つまり、基本的には論文の概要、第1章、終章、本文はバラバラに読まれる。特に
論文の概要は、概要のみを読んで終わりにすることも多いことから、論文の内容
すべてが概要において書かれている必要がある。また、論文の概要だけで完結し
ている(本文や参考文献を読まずともよい)ことが重要となる。

\begin{flushleft}
 {\bf チェックリスト}
\end{flushleft}
\begin{list}%
 {□} %default label
 {} %formatting parameter
 \item 指定された字数(ワード数、ページ数、行数)で書かれている。情報シ
       ステム工学科/コースでは、特に制限をしていないが A4 縦 1 ページ〜
       2 ページ程度にまとめること。
 \item 参考文献を引いていない \cite{R.A.Day_HowToWrite_Ja_98}。
 \item 数式を使っていない。
 \item 図や表を使っていない。
 \item 自分が行ったことについては過去時制で書かれている
       \cite{R.A.Day_HowToWrite_Ja_98}。
 \item 背景が簡潔に書いてある。
 \item 取り組んだ問題が書いてある \cite{Sakai_HowToWrite_02}。
 \item 着眼点が書いてある \cite{Sakai_HowToWrite_02}。
 \item 研究対象が書いてある \cite{Sakai_HowToWrite_02}。
 \item 研究手法が書いてある \cite{Sakai_HowToWrite_02}。
 \item 研究結果が書いてある \cite{Sakai_HowToWrite_02}。
 \item 結論が書いてある \cite{Sakai_HowToWrite_02}。
 \item 論文の構成が書いてある。
\end{list}


\subsection{第1章・はじめに}

この部分で研究の必要性と重要性、研究の目的を説明する。

\begin{flushleft}
 {\bf チェックリスト}
\end{flushleft}
\begin{list}%
 {□} %default label
 {} %formatting parameter
 \item 造語や略語の定義を行っている。
 \item 取り組む問題が書いてある。
 \item どうしてその問題に取り組むべきなのかが書いてある。
 \item どういう着眼点でその問題に取り組むのかが書いてある。
 \item 実際に何を行うのかが書いてある。
 \item 論文の構成が書いてある。
 \item 他人が行ったことと自分が行ったこととが混ざっていない。つまり、あ
       る段落から前は他人が行ったこと、その段落から後は自分が行ったこと
       が書いてある。
 \item 卒論テーマが研究室プロジェクトの一環であるとき、論文の背景がプロ
       ジェクト自体の背景ではなく、自分の研究テーマの直接的な背景になっ
       ている。
\end{list}


\subsection{終章・おわりに}

研究の目的に対してどのくらいまで貢献できたのかをまとめる。また、今後の課
題を述べる。

\begin{flushleft}
 {\bf まとめに対するチェックリスト}
\end{flushleft}
\begin{list}%
 {□} %default label
 {} %formatting parameter
 \item この研究で何をおこなったのかをまとめている。
 \item 研究の結果、研究目的をどの程度達成できたのかを述べている。
 \item 研究の成果がどれぐらいの学術的あるいは工学的価値があるのかを述べ
       ている。
\end{list}


\begin{flushleft}
 {\bf 今後の課題に対するチェックリスト}
\end{flushleft}
\begin{list}%
 {□} %default label
 {} %formatting parameter
 \item 研究目的を完全達成するためには何を行わなければならないかを述べて
       いる。
 \item 研究成果を生かして発展的な研究として何ができるかを述べている。
\end{list}


\subsection{他の章/節に関するチェックリスト}

\begin{flushleft}
 {\bf チェックリスト}
\end{flushleft}
\begin{list}%
 {□} %default label
 {} %formatting parameter
 \item 新しいシステムを設計し実現する際のシステムへの要求が「〜のため
       に、… なければならない」という形で記述されている。
 \item 実験の章が、目的、計測対象、前提条件、計測方法、結果の順番にまと
       められている。
 \item 結果の章には、実験方法の概略と実験結果、そして、その実験結果を簡
       単にまとめた説明が書かれている。
 \item 考察の章には、結果の概略と結果に基づいた主張が書かれている。
\end{list}

\section{7回目のチェック}

論文全体の流れをチェックする。

\begin{flushleft}
 {\bf チェックリスト}
\end{flushleft}
\begin{list}%
 {□} %default label
 {} %formatting parameter
 \item タイトルと研究目的、研究成果、結論が対応している。
 \item 頭とお尻が一致している。第1章で提示された問題が終章で解決か緩和さ
       れている。
 \item 比較対象の欠点を列挙したならば、自分の提案手法で列挙した欠点すべ
       てを解決している。あるいは、提案手法で解決できない欠点については
       考察や今後の課題で言及している。
 \end{list}

\section{指導を受けるための準備}

第1稿ができあがったら、指導教員や指導してくれている先輩に見てもらう。そ
の際にも、相手が指導しやすいように配慮するのが重要である。程研究室では、
ソフトウェア開発系の論文については、ソースコードを付録につけることになっ
ているが、論文指導を受ける段階では特に指示が無い限りソースコードを付録と
してつけなくても良い。

\begin{flushleft}
 {\bf チェックリスト}
\end{flushleft}
\begin{list}%
 {□} %default label
 {} %formatting parameter
 \item あらかじめ決められていた提出期限を守れている。 
 \item 論文は片面印刷かつダブルスペース(行間を広くあけてある。LaTeXならば\\
       「\verb+\setlength{\baselineskip}{1.6\baselineskip}+」で行間を操作で
       きる)で印刷されている。
 \item 論文は左肩止めになっている。
 \item 渡した論文のバージョン、誰に渡したのかが区別できるようになってい
       る。
 \end{list}

第2稿目以降に指導を受ける場合は以下もチェックすること。

\begin{flushleft}
 {\bf チェックリスト}
\end{flushleft}
\begin{list}%
 {□} %default label
 {} %formatting parameter
 \item 前回指導された論文の原稿も一緒に提出している。
 \item 前回指摘された点がすべて直っている。
 \item 前回指摘された点についての指示や助言に従っていない場合は、その理
       由が説明してある。
 \end{list}

\section{おわりに}

このチェックリストだけで良い論文がかけるようになるわけはないので、参考文
献に列挙してある論文の書き方に関する本を読んで、より知識を増やしてほしい。
むしろ、可能な限り卒業論文や修士論文を書く前に、論文の書き方に関する本を
数冊読んでおいて欲しい。また、研究室の先輩達の論文も読んでおいてほしい。

\begin{thebibliography}{99}

 \bibitem{Wikipedia.en_Psuedocode} 
Wikipedia.en: Pseudocode, 
	 \url{http://en.wikipedia.org/w/index.php?title=Pseudocode&oldid=346137702},
	 Feb 2010.

 \bibitem{Sakai_HowToWrite_02} 
酒井 聡樹: これから論文を書く若者のために大改定増補版, 共立出版, 2002年.

 \bibitem{IPSJ_HowToWrite_ver7.1} 
中島 浩, 斉藤 康己: LaTeXによる論文作成ガイド(第7.1版), 情報処理学会, 2009年.

 \bibitem{R.A.Day_HowToWrite_Ja_98}
ロバート デイ 著, 美宅 成樹 訳: はじめての科学英語論文 第2版, 丸善株式会社, 2001年.

 \bibitem{R.M.Lewis_HowToWrite_Ja_04} 
ロバート M.\ ルイス, エバン R.\ ホイットビー, ナンシー L.\ ホイットビー:
	 科学者・技術者のための英語論文の書き方−国際的に通用する論文を書く秘訣, 東京化学同人, 2004年.
\end{thebibliography}
\end{document}